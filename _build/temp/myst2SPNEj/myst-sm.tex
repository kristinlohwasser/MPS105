\section{Part 2b: The Standard Model of Particle Physics}

\subsection{Particles and forces: a brief introduction to the Standard Model}

\subsubsection{The particles of the Standard Model}

There are two types of fundamental particles: Matter particles (so-called fermions) and force particles (bosons).

the full manifold of elementary particles, only few are met ``in the everyday life''. These
are u and d quarks in the form of protons (udd) and neutrons (uud), electrons and,
out of interaction carriers, the photon. The reasons for that are different for different
particles. In particular, neutrino does not interact with the electromagnetic field and
is therefore very hard to detect; heavy particles are unstable and decay to lighter ones;
strongly interacting quarks and gluons are confined in hadrons. The full manyfold of SM
particles reveal themselves either in complicated dedicated experiments, or indirectly
by their effect seen in astrophysical observations

\paragraph{Fermions: matter particles}

The matter particles can be divided again into two classes: leptons and quarks. Leptons are either charged (electron, muon or tau) or neutral (all neutrinos). Quarks have non-integer charge of +2/3 or -1/3.  Every matter particle has an anti-particle with the same mass but opposite charge. This means, there are 24 matter particles within the Standard Model. Fermions can be grouped into so-called generations with the particles in the larger generations having the same properties (such as charge) but a larger mass. For example, the electron, muons and taus have the same charge, interact with the same forces and the same coupling strength but whilst the electron mass is 0.511 MeV, the muon mass is 103 MeV and the tau has a mass of 1777 MeV.

\textit{Table of Fermions}

\bigskip\noindent
\begin{tabular}{p{\dimexpr 0.200\linewidth-2\tabcolsep}p{\dimexpr 0.200\linewidth-2\tabcolsep}p{\dimexpr 0.200\linewidth-2\tabcolsep}p{\dimexpr 0.200\linewidth-2\tabcolsep}p{\dimexpr 0.200\linewidth-2\tabcolsep}}
\toprule
Fermion & Charge & Generation 1 & Generation 2 & Generation 3 \\
\hline
Leptons & -1 & electron & muon & tau \\
Leptons & 0 & electron-neutrino & muon-neutrino & tau-neutrino \\
Quarks & +2/3 & up & charm & top \\
Quarks & -1/3 & down & strange & bottom \\
\bottomrule
\end{tabular}

\bigskip\paragraph{Bosons: force carriers}

In the Standard Model, the action of forces is explained as the result of exchanges of force particles, which transfer energy  and momentum (and sometimes other properties such as charge) between the interacting particles.  The Standard Model describes three fundamental forces, each of which has its own exchange particle or particles.

\textit{Table of Bosons:}

\bigskip\noindent
\begin{tabular}{p{\dimexpr 0.250\linewidth-2\tabcolsep}p{\dimexpr 0.250\linewidth-2\tabcolsep}p{\dimexpr 0.250\linewidth-2\tabcolsep}p{\dimexpr 0.250\linewidth-2\tabcolsep}}
\toprule
Boson & Charge & Mass & Force \\
\hline
Photon & 0 & 0 & electromagnetic \\
W Boson & +/-1 & 80.4 GeV & weak (charge current) \\
Z Boson & 0 & 91.2 GeV & weak (neutral current) \\
gluon & 0 & 0 & strong force \\
Higgs & 0 & 125 GeV & gives rise to masses of other particles (not a force strictly speaking) \\
\bottomrule
\end{tabular}

\bigskip\bigskip\noindent
\begin{tabular}{p{\dimexpr 0.200\linewidth-2\tabcolsep}p{\dimexpr 0.200\linewidth-2\tabcolsep}p{\dimexpr 0.200\linewidth-2\tabcolsep}p{\dimexpr 0.200\linewidth-2\tabcolsep}p{\dimexpr 0.200\linewidth-2\tabcolsep}}
\toprule
Force & Particle & Strength & Range & Affects \\
\hline
Strong & gluon & 1 & 10$^{ -15}$ m & quarks \\
electromagnetic & photon & 1/137 & infinite & charged particles \\
weak & W,Z Bosons & 10$^{ -6}$ & 10$^{ -18}$ m & everything \\
\bottomrule
\end{tabular}

\bigskipThe following pictures summarized the Standard Model, giving the mass, charge and the spin of each particle. The spin of a particle is a quantum mechanical property that behaves like an intrinsic angular momentum. Particles that have half-integer spin are called fermions and obey the Pauli exclusion principle (no two fermions can have the same quantum numbers in the same place at the same time); particles that have integer spin are bosons and do not obey the exclusion principle (for example, you can have any number of identical photons).

It also highlights different groupings of particles (in the lecture slides, these are given separately), namely:

\begin{itemize}
\item fermions
\item quarks
\item leptons
\item 1st generation
\item 2nd generation
\item 3rd generation
\item all bosons
\item Force carriers: Bosons with Spin 1
\item particles participating in the strong interaction
\item particles participating in the electromagnetic interacton
\item particles participating in the weak interacton
\item Higgs Boson that gives mass to all fermions and heavy weak bosons
\item all particles
\end{itemize}

\includegraphics[width=0.7\linewidth]{files/SM_particles_slides1-f870cb29f5d1a83b23bbbb6076bcceee.png}

\paragraph{Quarks}\label{sec_quarks}

No interaction in the Standard Model can convert a quark into a lepton or vice versa. There exist conservation laws describing this fact.

Baryon number is defined as the total number of baryons minus the total number of anti-baryons. Similarly, the number of leptons is conserved: Each lepton genera­tion has its own conserved lepton number.

\paragraph{Open questions of the Standard Model}

\subsubsection{Units}

Particle physics uses a specific system of units which is designed to make calculations easier - This is in contrast to the system of units that is normally used within physics, the International System of Units (SI) which is metric and what you have been dealing with so far (and in future too).

Energy is generally measured in electron-Volt (eV). An electron volt is defined as the amount of kinetic energy gained by a single electron after being accelerated from rest through an electric potential difference of 1 Volt in vacuum:

\begin{equation}
E_{\textrm{kin}} = q \times V
\end{equation}

where $E_{\textrm{kin}}$ is the kinetic energy of the electron, $q$ is its charge and $V$ is the voltage (or voltage difference) with which it is accelerated. CHECK FOR REFERENCE TO 1st YEAR COURSE!
Given this definition, 1 eV $= 1.602 \times 10^{ -19}$ J. So to convert from Joule into eV, one has to divide by $1.602 \times 10^{ -19}$.

In the system of units used in particle physics, the speed of light is set to $c=1$ and also $\hbar = 1$. These are so-called `natural units'. This allows to have the same units for energy, momentum and mass, which simplifies calculations as there is no factors of $c^2$ to numerically take into account.

The barn is a unit of area which is used in cross-sections and integrated luminosity. One barn is equal to 10$^{ -28}$ m$^2$ or 10$^{ -24}$ cm$^2$. This is a larger area, common are cross-section smaller than mb or nb (millibarns and nanobarns).

\paragraph{General properties of particles: mass, charge and spin}

The invariant mass is defined as:

$m c^2 = \sqrt{E^2 - p^2c^2} = \sqrt{E^2 - p_x^2c^2 - p_y^2c^2 -p_z^2c^2}$

\paragraph{Conservation Laws}

\begin{framed}
\textbf{Calculation}\\
Consider two quarks at a distance of 1.0 fm which are are attracting each other with a force of magnitude 2.0 $\times$ 10$^4$ N. What is the magnitude of the force at a distance of 9.0 fm?

As the force is proprotional to $r$, the magnitude of the force at 9.0 fm is 9 $\times$ the force at 1.0 fm, thus 18.0 $\times$ 10$^4$ N.
\end{framed}